% Options for packages loaded elsewhere
\PassOptionsToPackage{unicode}{hyperref}
\PassOptionsToPackage{hyphens}{url}
%
\documentclass[
]{book}
\title{Interactive maps with leaflet}
\author{Rex Parsons}
\date{2022-05-22}

\usepackage{amsmath,amssymb}
\usepackage{lmodern}
\usepackage{iftex}
\ifPDFTeX
  \usepackage[T1]{fontenc}
  \usepackage[utf8]{inputenc}
  \usepackage{textcomp} % provide euro and other symbols
\else % if luatex or xetex
  \usepackage{unicode-math}
  \defaultfontfeatures{Scale=MatchLowercase}
  \defaultfontfeatures[\rmfamily]{Ligatures=TeX,Scale=1}
\fi
% Use upquote if available, for straight quotes in verbatim environments
\IfFileExists{upquote.sty}{\usepackage{upquote}}{}
\IfFileExists{microtype.sty}{% use microtype if available
  \usepackage[]{microtype}
  \UseMicrotypeSet[protrusion]{basicmath} % disable protrusion for tt fonts
}{}
\makeatletter
\@ifundefined{KOMAClassName}{% if non-KOMA class
  \IfFileExists{parskip.sty}{%
    \usepackage{parskip}
  }{% else
    \setlength{\parindent}{0pt}
    \setlength{\parskip}{6pt plus 2pt minus 1pt}}
}{% if KOMA class
  \KOMAoptions{parskip=half}}
\makeatother
\usepackage{xcolor}
\IfFileExists{xurl.sty}{\usepackage{xurl}}{} % add URL line breaks if available
\IfFileExists{bookmark.sty}{\usepackage{bookmark}}{\usepackage{hyperref}}
\hypersetup{
  pdftitle={Interactive maps with leaflet},
  pdfauthor={Rex Parsons},
  hidelinks,
  pdfcreator={LaTeX via pandoc}}
\urlstyle{same} % disable monospaced font for URLs
\usepackage{color}
\usepackage{fancyvrb}
\newcommand{\VerbBar}{|}
\newcommand{\VERB}{\Verb[commandchars=\\\{\}]}
\DefineVerbatimEnvironment{Highlighting}{Verbatim}{commandchars=\\\{\}}
% Add ',fontsize=\small' for more characters per line
\usepackage{framed}
\definecolor{shadecolor}{RGB}{248,248,248}
\newenvironment{Shaded}{\begin{snugshade}}{\end{snugshade}}
\newcommand{\AlertTok}[1]{\textcolor[rgb]{0.94,0.16,0.16}{#1}}
\newcommand{\AnnotationTok}[1]{\textcolor[rgb]{0.56,0.35,0.01}{\textbf{\textit{#1}}}}
\newcommand{\AttributeTok}[1]{\textcolor[rgb]{0.77,0.63,0.00}{#1}}
\newcommand{\BaseNTok}[1]{\textcolor[rgb]{0.00,0.00,0.81}{#1}}
\newcommand{\BuiltInTok}[1]{#1}
\newcommand{\CharTok}[1]{\textcolor[rgb]{0.31,0.60,0.02}{#1}}
\newcommand{\CommentTok}[1]{\textcolor[rgb]{0.56,0.35,0.01}{\textit{#1}}}
\newcommand{\CommentVarTok}[1]{\textcolor[rgb]{0.56,0.35,0.01}{\textbf{\textit{#1}}}}
\newcommand{\ConstantTok}[1]{\textcolor[rgb]{0.00,0.00,0.00}{#1}}
\newcommand{\ControlFlowTok}[1]{\textcolor[rgb]{0.13,0.29,0.53}{\textbf{#1}}}
\newcommand{\DataTypeTok}[1]{\textcolor[rgb]{0.13,0.29,0.53}{#1}}
\newcommand{\DecValTok}[1]{\textcolor[rgb]{0.00,0.00,0.81}{#1}}
\newcommand{\DocumentationTok}[1]{\textcolor[rgb]{0.56,0.35,0.01}{\textbf{\textit{#1}}}}
\newcommand{\ErrorTok}[1]{\textcolor[rgb]{0.64,0.00,0.00}{\textbf{#1}}}
\newcommand{\ExtensionTok}[1]{#1}
\newcommand{\FloatTok}[1]{\textcolor[rgb]{0.00,0.00,0.81}{#1}}
\newcommand{\FunctionTok}[1]{\textcolor[rgb]{0.00,0.00,0.00}{#1}}
\newcommand{\ImportTok}[1]{#1}
\newcommand{\InformationTok}[1]{\textcolor[rgb]{0.56,0.35,0.01}{\textbf{\textit{#1}}}}
\newcommand{\KeywordTok}[1]{\textcolor[rgb]{0.13,0.29,0.53}{\textbf{#1}}}
\newcommand{\NormalTok}[1]{#1}
\newcommand{\OperatorTok}[1]{\textcolor[rgb]{0.81,0.36,0.00}{\textbf{#1}}}
\newcommand{\OtherTok}[1]{\textcolor[rgb]{0.56,0.35,0.01}{#1}}
\newcommand{\PreprocessorTok}[1]{\textcolor[rgb]{0.56,0.35,0.01}{\textit{#1}}}
\newcommand{\RegionMarkerTok}[1]{#1}
\newcommand{\SpecialCharTok}[1]{\textcolor[rgb]{0.00,0.00,0.00}{#1}}
\newcommand{\SpecialStringTok}[1]{\textcolor[rgb]{0.31,0.60,0.02}{#1}}
\newcommand{\StringTok}[1]{\textcolor[rgb]{0.31,0.60,0.02}{#1}}
\newcommand{\VariableTok}[1]{\textcolor[rgb]{0.00,0.00,0.00}{#1}}
\newcommand{\VerbatimStringTok}[1]{\textcolor[rgb]{0.31,0.60,0.02}{#1}}
\newcommand{\WarningTok}[1]{\textcolor[rgb]{0.56,0.35,0.01}{\textbf{\textit{#1}}}}
\usepackage{longtable,booktabs,array}
\usepackage{calc} % for calculating minipage widths
% Correct order of tables after \paragraph or \subparagraph
\usepackage{etoolbox}
\makeatletter
\patchcmd\longtable{\par}{\if@noskipsec\mbox{}\fi\par}{}{}
\makeatother
% Allow footnotes in longtable head/foot
\IfFileExists{footnotehyper.sty}{\usepackage{footnotehyper}}{\usepackage{footnote}}
\makesavenoteenv{longtable}
\usepackage{graphicx}
\makeatletter
\def\maxwidth{\ifdim\Gin@nat@width>\linewidth\linewidth\else\Gin@nat@width\fi}
\def\maxheight{\ifdim\Gin@nat@height>\textheight\textheight\else\Gin@nat@height\fi}
\makeatother
% Scale images if necessary, so that they will not overflow the page
% margins by default, and it is still possible to overwrite the defaults
% using explicit options in \includegraphics[width, height, ...]{}
\setkeys{Gin}{width=\maxwidth,height=\maxheight,keepaspectratio}
% Set default figure placement to htbp
\makeatletter
\def\fps@figure{htbp}
\makeatother
\setlength{\emergencystretch}{3em} % prevent overfull lines
\providecommand{\tightlist}{%
  \setlength{\itemsep}{0pt}\setlength{\parskip}{0pt}}
\setcounter{secnumdepth}{5}
\usepackage{booktabs}
\usepackage{amsthm}
\makeatletter
\def\thm@space@setup{%
  \thm@preskip=8pt plus 2pt minus 4pt
  \thm@postskip=\thm@preskip
}
\makeatother
\ifLuaTeX
  \usepackage{selnolig}  % disable illegal ligatures
\fi
\usepackage[]{natbib}
\bibliographystyle{apalike}

\begin{document}
\maketitle

{
\setcounter{tocdepth}{1}
\tableofcontents
}
\hypertarget{prerequisites}{%
\chapter{Prerequisites}\label{prerequisites}}

This book is intended as a non-comprehensive guide to developing interactive maps with leaflet and shiny. This is by no means comprehensive as it is based on methods that were used in developing the\href{https://access.healthequity.link/}{iTRAQI shiny app}. However, since this book does focus on the applied problem of developing the iTRAQI shiny app, it includes specific help and methods for these are described here that may be otherwise difficult to find.

For a more comprehensive introduction to leaflet, see the
\href{https://rstudio.github.io/leaflet/}{leaflet documentation}.

For a more comprehensive introduction to shiny, see the
\href{https://mastering-shiny.org/}{Mastering Shiny book}

\hypertarget{intro}{%
\chapter{Introduction}\label{intro}}

This book focuses on using \texttt{leaflet} and \texttt{shiny} together to make interactive maps.

Here's a simple leaflet map.

\begin{Shaded}
\begin{Highlighting}[]
\FunctionTok{library}\NormalTok{(leaflet)}

\FunctionTok{leaflet}\NormalTok{() }\SpecialCharTok{\%\textgreater{}\%}
  \FunctionTok{addTiles}\NormalTok{() }\SpecialCharTok{\%\textgreater{}\%}  \CommentTok{\# Add default OpenStreetMap map tiles}
  \FunctionTok{addMarkers}\NormalTok{(}\AttributeTok{lng=}\FloatTok{174.768}\NormalTok{, }\AttributeTok{lat=}\SpecialCharTok{{-}}\FloatTok{36.852}\NormalTok{, }\AttributeTok{popup=}\StringTok{"The birthplace of R"}\NormalTok{)}
\end{Highlighting}
\end{Shaded}

\begin{figure}
\includegraphics[width=1\linewidth]{interactive-maps_files/figure-latex/leaflet-simple-1} \caption{Simple leaflet map}\label{fig:leaflet-simple}
\end{figure}

Before we begin adding to this map, we need to create the layers that we want to add.

In the iTRAQI app, we used markers, rasters and polygons to show the key locations and interpolations.

See the \href{https://access.healthequity.link/}{iTRAQI shiny app here} and read more about it in the information tab of the app.

Chapter \ref{building} will focus on these first steps, before making any maps or interactivity. If you're already well-versed in making these layers and the \texttt{sf} R package, you can skip to the latter chapters.

\hypertarget{leaflet-layers}{%
\section{leaflet layers}\label{leaflet-layers}}

\begin{itemize}
\item
  To display statistical area level 1 and 2 (SA1 and SA2) regions on the map, we will be using \texttt{sf} objects with MULTIPOLYGON geometries. These are multipolygons because some of these areas include distinct areas, such as a set of islands, that aren't contained within a single polygon.
\item
  To display the location of acute and rehab centers and town locations with travel times that we used for interpolations, we used (spatial) data.frames that had longitudes and latitudes for their location.
\item
  To display the continuous interpolations, we used \href{https://rdrr.io/cran/raster/man/raster.html}{\texttt{RasterLayer}} objects.
\end{itemize}

Using a polygon and raster layer that's used in the iTRAQI map and some markers in a data.frame, we can make see the basic approach that we use to display these on a leaflet map.

First, lets make a data.frame with the coordinates for the Princess Alexandra and Townsville University Hospitals, and download a raster and polygon layer from the iTRAQI app GitHub repository.

\begin{Shaded}
\begin{Highlighting}[]
\FunctionTok{library}\NormalTok{(tidyverse)}
\end{Highlighting}
\end{Shaded}

\begin{verbatim}
## -- Attaching packages --------------------------------------- tidyverse 1.3.1 --
\end{verbatim}

\begin{verbatim}
## v ggplot2 3.3.6     v purrr   0.3.4
## v tibble  3.1.2     v dplyr   1.0.6
## v tidyr   1.1.3     v stringr 1.4.0
## v readr   2.1.2     v forcats 0.5.1
\end{verbatim}

\begin{verbatim}
## -- Conflicts ------------------------------------------ tidyverse_conflicts() --
## x dplyr::filter() masks stats::filter()
## x dplyr::lag()    masks stats::lag()
\end{verbatim}

\begin{Shaded}
\begin{Highlighting}[]
\FunctionTok{library}\NormalTok{(sf)}
\end{Highlighting}
\end{Shaded}

\begin{verbatim}
## Linking to GEOS 3.9.1, GDAL 3.2.1, PROJ 7.2.1; sf_use_s2() is TRUE
\end{verbatim}

\begin{Shaded}
\begin{Highlighting}[]
\NormalTok{download\_layer }\OtherTok{\textless{}{-}} \ControlFlowTok{function}\NormalTok{(layer\_name, }\AttributeTok{save\_dir=}\StringTok{"input"}\NormalTok{) \{}
\NormalTok{  githubURL }\OtherTok{\textless{}{-}}\NormalTok{ glue}\SpecialCharTok{::}\FunctionTok{glue}\NormalTok{(}\StringTok{"https://raw.githubusercontent.com/RWParsons/iTRAQI\_app/main/input/layers/\{layer\_name\}"}\NormalTok{)}
  \FunctionTok{download.file}\NormalTok{(githubURL, }\FunctionTok{file.path}\NormalTok{(save\_dir, layer\_name), }\AttributeTok{method=}\StringTok{"curl"}\NormalTok{)}
  \FunctionTok{readRDS}\NormalTok{(}\FunctionTok{file.path}\NormalTok{(save\_dir, layer\_name))}
\NormalTok{\}}

\NormalTok{raster\_layer }\OtherTok{\textless{}{-}} \FunctionTok{download\_layer}\NormalTok{(}\StringTok{"rehab\_raster.rds"}\NormalTok{) }\SpecialCharTok{\%\textgreater{}\%}
\NormalTok{  raster}\SpecialCharTok{::}\FunctionTok{raster}\NormalTok{(., }\AttributeTok{layer=}\DecValTok{1}\NormalTok{)}

\NormalTok{polygons\_layer }\OtherTok{\textless{}{-}} \FunctionTok{download\_layer}\NormalTok{(}\StringTok{"stacked\_SA1\_and\_SA2\_polygons\_year2016\_simplified.rds"}\NormalTok{)}
\NormalTok{polygons\_layer }\OtherTok{\textless{}{-}}\NormalTok{ polygons\_layer[polygons\_layer}\SpecialCharTok{$}\NormalTok{SA\_level}\SpecialCharTok{==}\DecValTok{2}\NormalTok{, ] }\CommentTok{\# show SA2 regions for example}

\NormalTok{marker\_locations }\OtherTok{\textless{}{-}} \FunctionTok{data.frame}\NormalTok{(}
  \AttributeTok{centre\_name=}\FunctionTok{c}\NormalTok{(}\StringTok{"Princess Alexandra Hospital (PAH)"}\NormalTok{, }\StringTok{"Townsville University Hospital"}\NormalTok{),}
  \AttributeTok{x=}\FunctionTok{c}\NormalTok{(}\FloatTok{153.033519}\NormalTok{, }\FloatTok{146.762041}\NormalTok{),}
  \AttributeTok{y=}\FunctionTok{c}\NormalTok{(}\SpecialCharTok{{-}}\FloatTok{27.497374}\NormalTok{, }\SpecialCharTok{{-}}\FloatTok{19.320502}\NormalTok{)}
\NormalTok{)}
\end{Highlighting}
\end{Shaded}

Here, in figure \ref{fig:leaflet-objects}, we make a leaflet map with the three object types. We will use these three functions, \texttt{addPolygons()}, \texttt{addRasterImage()}, and \texttt{addMarkers()} to add almost all of the content to our leaflet maps.

\begin{Shaded}
\begin{Highlighting}[]
\FunctionTok{leaflet}\NormalTok{() }\SpecialCharTok{\%\textgreater{}\%}
  \FunctionTok{addProviderTiles}\NormalTok{(}\StringTok{"CartoDB.VoyagerNoLabels"}\NormalTok{) }\SpecialCharTok{\%\textgreater{}\%} \CommentTok{\# add a simple base map}
  \FunctionTok{addPolygons}\NormalTok{(}
    \AttributeTok{data=}\NormalTok{polygons\_layer,}
    \AttributeTok{fillColor=}\StringTok{"Orange"}\NormalTok{,}
    \AttributeTok{color=}\StringTok{"black"}\NormalTok{,}
    \AttributeTok{weight=}\DecValTok{1}\NormalTok{,}
    \AttributeTok{group=}\StringTok{"Polygons"}
\NormalTok{  ) }\SpecialCharTok{\%\textgreater{}\%}
  \FunctionTok{addRasterImage}\NormalTok{(}
    \AttributeTok{x=}\NormalTok{raster\_layer,}
    \AttributeTok{colors=}\StringTok{"YlOrRd"}\NormalTok{,}
    \AttributeTok{group=}\StringTok{"Raster"}
\NormalTok{  ) }\SpecialCharTok{\%\textgreater{}\%}
  \FunctionTok{addMarkers}\NormalTok{(}
    \AttributeTok{lng=}\NormalTok{marker\_locations}\SpecialCharTok{$}\NormalTok{x, }
    \AttributeTok{lat=}\NormalTok{marker\_locations}\SpecialCharTok{$}\NormalTok{y,}
    \AttributeTok{label=}\NormalTok{marker\_locations}\SpecialCharTok{$}\NormalTok{centre\_name,}
    \AttributeTok{group=}\StringTok{"Points"}
\NormalTok{  ) }\SpecialCharTok{\%\textgreater{}\%}
  \FunctionTok{addLayersControl}\NormalTok{(}
    \AttributeTok{position=}\StringTok{"topright"}\NormalTok{,}
    \AttributeTok{baseGroups=}\FunctionTok{c}\NormalTok{(}\StringTok{"Polygons"}\NormalTok{, }\StringTok{"Raster"}\NormalTok{, }\StringTok{"Points"}\NormalTok{),}
    \AttributeTok{options=}\FunctionTok{layersControlOptions}\NormalTok{(}\AttributeTok{collapsed =} \ConstantTok{FALSE}\NormalTok{)}
\NormalTok{  ) }
\end{Highlighting}
\end{Shaded}

\begin{figure}
\includegraphics[width=1\linewidth]{interactive-maps_files/figure-latex/leaflet-objects-1} \caption{leaflet map with polygons, rasters and markers}\label{fig:leaflet-objects}
\end{figure}

Almost all of these objects were made before being used in the shiny app. Chapter \ref{building} will introduce the methods used to make them. Chapter \ref{shiny-intro} will introduce the basics of a shiny app. Chapter \ref{shiny-methods} will introduce the more specific methods that were used to construct the iTRAQI app itself.

\hypertarget{building}{%
\chapter{Creating the layers}\label{building}}

This chapter will cover the necessary steps to make layers which will be visualised in the app:

\begin{itemize}
\tightlist
\item
  kriging
\item
  spatial joins
\item
  aggregating interpolations within polygons
\end{itemize}

\hypertarget{shiny-intro}{%
\chapter{An app with a map}\label{shiny-intro}}

This chapter will have a brief intro to shiny with a map:

\begin{itemize}
\tightlist
\item
  ui and server
\item
  reactivity
\item
  leaflet
\item
  leafletproxy
\end{itemize}

\hypertarget{shiny-methods}{%
\chapter{An app with a map}\label{shiny-methods}}

This chapter will have a brief intro to shiny with a map:

\begin{itemize}
\tightlist
\item
  ui and server
\item
  reactivity
\item
  leaflet
\item
  leafletproxy
\end{itemize}

  \bibliography{book.bib,packages.bib}

\end{document}
